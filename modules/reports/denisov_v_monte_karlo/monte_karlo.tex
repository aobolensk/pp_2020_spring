\documentclass{report}

\usepackage[english, russian]{babel}
\usepackage[pdftex]{hyperref}
\usepackage[14pt]{extsizes}
\usepackage[numbers,sort&compress]{natbib}
\usepackage{amsmath}
\usepackage{amsfonts}
\usepackage{alltt}
\usepackage{listings}
\usepackage{color}
\usepackage{geometry}
\usepackage{enumitem}
\usepackage{multirow}
\usepackage{graphicx}

% Устанавливаю поля
\geometry{a4paper,top=2cm,bottom=3cm,left=2cm,right=1.5cm}
\setlength{\parskip}{0.5cm}
\setlist{nolistsep, itemsep=0.3cm,parsep=0pt}

% Мой стиль для листинга кода
\lstset{language=C++,
		basicstyle=\footnotesize,
		keywordstyle=\color{blue}\ttfamily,
		stringstyle=\color{red}\ttfamily,
		commentstyle=\color{green}\ttfamily,
		morecomment=[l][\color{magenta}]{\#}, 
		tabsize=4
}

% Замена [1] -> 1. в списке литературы
\makeatletter
\renewcommand\@biblabel[1]{#1.\hfil}
\makeatother

\begin{document}

% Титульный лист
\begin{titlepage}

\begin{center}
Министерство науки и высшего образования Российской Федерации
\end{center}
\begin{center}
Федеральное государственное автономное образовательное учреждение высшего образования
\end{center}
\begin{center}
Национальный исследовательский Нижегородский государственный университет им. Н.И. Лобачевского
\end{center}
\begin{center}
Институт информационных технологий, математики и механики
\end{center}

\vspace{4em}

\begin{center}
\textbf{Отчет по лабораторной работе} \\
\end{center}
\begin{center}
\textbf{«Вычисление многомерных интегралов методом Монте-Карло»} \\
\end{center}

\vspace{4em}

\newbox{\lbox}
\savebox{\lbox}{\hbox{text}}
\newlength{\maxl}
\setlength{\maxl}{\wd\lbox}
\hfill\parbox{7cm}{
\hspace*{5cm}\hspace*{-5cm}\textbf{Выполнил:} \\ студент группы 381706-1 \\ Денисов В. Л.\\
\\
\hspace*{5cm}\hspace*{-5cm}\textbf{Проверил:}\\ доцент кафедры МОСТ, \\ кандидат технических наук \\ Сысоев А. В.
}

\vspace{\fill}

\begin{center} Нижний Новгород \\ 2020 \end{center}

\end{titlepage}
% Конец титульного листа

\setcounter{page}{2}

% Содержание
\tableofcontents
\newpage

% Введение
\section*{Введение}
\addcontentsline{toc}{section}{Введение}
\indent\par Вычисление многомерных интегралов - непростая задача.
\par Новый абзац
\newpage
% Конец введения

% Постановка задачи
\section*{Постановка задачи}
\addcontentsline{toc}{section}{Постановка задачи}
\indent\par В рамках лабораторной работы ставится задача разработки нескольких библиотек, реализующих алгоритм метода Монте-Карло для вычисления многомерных интегралов.
\par В итоге программный комплекс должен поддерживать:
\begin{itemize}
\item Последовательный алгоритм метода Монте-Карло;
\item Параллельный алгоритм метода Монте-Карло при помощи технологий OpenMP, TBB, std::threads;
\end{itemize}
\par Программное решение будет представлено следующим образом:
\begin{enumerate} 
\itemОтдельный модуль для каждой библиотеки, реализующей свою технологию.
\itemНабор автоматических тестов с использованием Google C++ Testing Framework для каждой из технологий.
\end{enumerate} 

\newpage
% Конец постановки задачи

% Метод решения
\section*{Метод решения}
\addcontentsline{toc}{section}{Метод решения}
Текст решения
\newpage
% Конец метода решения

% Схема распараллеливания
\section*{Схема распараллеливания}
\addcontentsline{toc}{section}{Схема распараллеливания}
Текст схемы распараллеливания
\newpage
% Конец схемы распараллеливания

% Описание программной реализации
\section*{Описание программной реализации}
\addcontentsline{toc}{section}{Описание программной реализации}

\subsection*{OpenMP}
\addcontentsline{toc}{subsection}{OpenMP}
Текст описания OpenMP

\subsection*{TBB}
\addcontentsline{toc}{subsection}{TBB}
Текст описания TBB

\subsection*{std::threads}
\addcontentsline{toc}{subsection}{std::threads}
Текст описания std::threads

\newpage
% Конец описания программной реализации

% Подтверждение корректности
\section*{Подтверждение корректности}
\addcontentsline{toc}{section}{Подтверждение корректности}
\indent\par Для подтверждения корректности в программе представлен набор тестов, разработанных с помощью использования Google C++ Testing Framework.
\par Набор представляет из себя тесты, которые можно разделить на 2 связанных по смыслу блока: проверка корректности входных данных, а также проверка корректности вычислений.

\par К первому смысловому блоку относятся следующие тесты: 
\begin{itemize}
\item проверка используемого числа точек~--- должно быть положительным числом;
\item наличие значений в указанных пределах интегрирования~--- не могут быть пустыми;
\item проверка на эквивалентность размерностей пределов интегрирования~--- должны иметь одинаковую размерность.
\end{itemize}
 
\par Ко второму смысловому блоку относятся:
\begin{itemize}
\item Вычисление интеграла для константной функции;
\item Вычисление интеграла для функции одной переменной;
\item Вычисление интеграла для функции двух переменных;
\item Вычисление интеграла для функции трёх переменных;
\item Сравнение результата вычислений параллельной версии с результатом, полученном при последовательном выполнении.
\end{itemize}

\par Успешное прохождение всех тестов доказывает корректность работы программного комплекса.

\newpage
% Конец подтверждения корректности


% Результаты экспериментов
\section*{Результаты экспериментов}
\addcontentsline{toc}{section}{Результаты экспериментов}
\indent\par Вычислительные эксперименты для оценки эффективности параллельного варианта метода Монте-Карло для вычисления многомерных интегралов проводились на оборудовании со следующей аппаратной конфигурацией:

\begin{itemize}
\item Процессор: Intel Core i5-7200U, 2700 MHz, ядер: 2;
\item Оперативная память: 6012 МБ (DDR4 SDRAM), 2400 MHz;
\item ОС: Microsoft Windows 10 Home, версия 10.0.18363 сборка 18363.778.
\end{itemize}

\par Результаты экспериментов представлены в Таблице 1.

\begin{table}[!h]
\caption{Резултаты вычислительных экспериментов}
\centering
\begin{tabular}{|c|c|c|c|c|c|c|c|}
\hline
\multirow{3}{*}
	{\begin{tabular}[c]{@{}c@{}}Кол-во\\ потоков\end{tabular}} & 
\multirow{2}{*}
	{\begin{tabular}[c]{@{}c@{}}Последовательный\\ алгоритм\end{tabular}} & 
\multicolumn{6}{c|}
	{Параллельный алгоритм}	\\ 
	\cline{3-8} & & 
	\multicolumn{2}{c|}{OpenMP} & 
	\multicolumn{2}{c|}{TBB} & 
	\multicolumn{2}{c|}{std::threads} 
	\\ \cline{2-8}
	& t, с		& t, с & speedup		& t, с & speedup		& t, с & speedup		\\ \hline
2   & time       & time1 & su1           & time2 & su2           & time3 & su3           \\ \hline
4   & time       & time1 & su1           & time2 & su2           & time3  & su3          \\ \hline
\end{tabular}
\end{table}

\par По данным, полученным в результате экспериментов, можно сделать вывод о том, что параллельный случай работает действительно быстрее, чем последовательный. Однако при увеличении числа потоков не наблюдается большого прироста скорости вычислений. Это объясняется ограничениями, которые возникают из характеристик оборудования.

\newpage
% Конец результатов экспериментов


% Заключение
\section*{Заключение}
\addcontentsline{toc}{section}{Заключение}
\indent\par В результате лабораторной работы были разработаны несколько библиотек, реализующих метод Монте-Карло для вычисления многомерных интегрвалов.
\par Основной задачей данной лабораторной работы была реализация параллельной версии алгоритма при помощи использования различных технологий: OpenMP, TBB, std::threads. Эта задача была успешно достигнута, о чем говорят результаты экспериментов, проведенных в ходе работы. Они показывают, что параллельные варианты работают действительно быстрее, чем последовательный. Помимо этого, выполнено их сравнение друг с другом и сделаны соответствующие выводы в разделе "Результаты экспериментов".
\par Кроме того, были разработаны и доведены до успешного выполнения тесты, созданные для данного программного проекта с использованием Google C++ Testing Framework и необходимые для подтверждения корректности работы программы.

\newpage
% Конец заключения

% Список литературы
\begin{thebibliography}{1}
\addcontentsline{toc}{section}{Список литературы}
\bibitem{Barkalov} Баркалов К.А. Методы параллельных вычислений. Н. Новгород: Изд-во Нижегородского госуниверситета им. Н.И. Лобачевского, 2011
\bibitem{Wiki1} Википедия: свободная электронная энциклопедия: на русском языке [Электронный ресурс] // URL: https://ru.wikipedia.org/wiki/ (дата обращения: dd.mm.yyyy)
\end{thebibliography}
\newpage
% Конец списка литературы

% Приложение
\section*{Приложение}
\addcontentsline{toc}{section}{Приложение}
Здесь будет листинг кода

\begin{lstlisting}[label=testl,caption=Test]
int main(void) // main routine
{
int i, j; // Initialisation of counters

// The code below prints the 3x3 matrix
for (i=0; i<3; ++i) {
for (j=0; j<3; ++j) printf("%5.1f", m[i*3+j]);
putchar('\n');
}

cblas_dgemv(CblasRowMajor, CblasNoTrans,
3, 3, 1.0, m, 3, x, 1, 0.0, y, 1);

// The code below prints the 3x3 matrix - result of multiplication
for (i=0; i<3; ++i) printf("%5.1f\n", y[i]);

return 0;
}
\end{lstlisting}

\begin{lstlisting}[label=test2,caption=Test2]
int main(void) // main routine
{
	int i, j; // Initialisation of counters

	cout << i  << j;

	return 0;
}
\end{lstlisting}


\end{document}
\documentclass{report}

\usepackage[english, russian]{babel}
\usepackage[pdftex]{hyperref}
\usepackage[14pt]{extsizes}
\usepackage[numbers,sort&compress]{natbib}
\usepackage{amsmath}
\usepackage{amsfonts}
\usepackage{alltt}
\usepackage{listings}
\usepackage{color}

% my style of the listings
\lstset{language=C++,
                basicstyle=\footnotesize,
                keywordstyle=\color{blue}\ttfamily,
                stringstyle=\color{red}\ttfamily,
                commentstyle=\color{green}\ttfamily,
                morecomment=[l][\color{magenta}]{\#}, 
		   tabsize=4
}

% Можно ли избавиться от такого подхода для титульника?
\newenvironment{changemargin}[2]{%
\begin{list}{}{%
\setlength{\topsep}{0pt}%
\setlength{\leftmargin}{#1}%
\setlength{\rightmargin}{#2}%
}%
\item[]}{\end{list}}
%конец подхода


% [1] -> 1. in the bibliography
\makeatletter
\renewcommand\@biblabel[1]{#1.\hfil}
\makeatother

\begin{document}

% Титульный лист
\begin{titlepage}
\begin{changemargin}{-1cm}{-1cm}
\begin{center}
Министерство науки и высшего образования Российской Федерации	\\
\end{center}
\begin{center}
Федеральное государственное автономное образовательное учреждение высшего образования \\
\end{center}
\begin{center}
Национальный исследовательский Нижегородский государственный университет им. Н.И. Лобачевского	\\
\end{center}

\begin{center}
Институт информационных технологий, математики и механики \\
\end{center}

\vspace{4em}

\begin{center}
\textbf{Отчет по лабораторной работе} \\
\end{center}
\begin{center}
\textbf{«Вычисление многомерных интегралов методом Монте-Карло»} \\
\end{center}

\vspace{4em}

\newbox{\lbox}
\savebox{\lbox}{\hbox{text}}
\newlength{\maxl}
\setlength{\maxl}{\wd\lbox}
\hfill\parbox{7cm}{
\hspace*{5cm}\hspace*{-5cm}\textbf{Выполнил:} \\ студент группы 381706-1 \\ Денисов В. Л.\\
\\
\hspace*{5cm}\hspace*{-5cm}\textbf{Проверил:}\\ доцент кафедры МОСТ, кандидат технических наук \\ Сысоев А. В.
}

\vspace{\fill}

\begin{center} Нижний Новгород \\ 2020 \end{center}

\end{changemargin}
\end{titlepage}

% Конец титульного листа

\setcounter{page}{2}

\tableofcontents
\newpage

\section*{Введение}
\addcontentsline{toc}{section}{Введение}
\indent\parВычисление многомерных интегралов - непростая задача.
\parНовый абзац
\newpage

\section*{Постановка задачи}
\addcontentsline{toc}{section}{Постановка задачи}
Текст постановки задачи
\newpage

\section*{Метод решения}
\addcontentsline{toc}{section}{Метод решения}
Текст решения
\newpage

\section*{Схема распараллеливания}
\addcontentsline{toc}{section}{Схема распараллеливания}
Текст схемы распараллеливания
\newpage

\section*{Описание программной реализации}
\addcontentsline{toc}{section}{Описание программной реализации}

\subsection*{OpenMP}
\addcontentsline{toc}{subsection}{OpenMP}
Текст описания OpenMP

\subsection*{TBB}
\addcontentsline{toc}{subsection}{TBB}
Текст описания TBB

\subsection*{std::threads}
\addcontentsline{toc}{subsection}{std::threads}
Текст описания std::threads

\newpage

\section*{Подтверждение корректности}
\addcontentsline{toc}{section}{Подтверждение корректности}
Текст подтверждения корректности
\newpage

\section*{Результаты экспериментов}
\addcontentsline{toc}{section}{Результаты экспериментов}
Текст результатов экспериментов
\newpage

\section*{Заключение}
\addcontentsline{toc}{section}{Заключение}
Текст заключения
\newpage

\begin{thebibliography}{1}
\addcontentsline{toc}{section}{Список литературы}
\bibitem{Barkalov} Баркалов К.А. Методы параллельных вычислений. Н. Новгород: Изд-во Нижегородского госуниверситета им. Н.И. Лобачевского, 2011
\bibitem{Wiki1} Википедия: свободная электронная энциклопедия: на русском языке [Электронный ресурс] // URL: https://ru.wikipedia.org/wiki/ (дата обращения: dd.mm.yyyy)
\end{thebibliography}
\newpage

\section*{Приложение}
\addcontentsline{toc}{section}{Приложение}
Здесь будет листинг кода

\begin{lstlisting}[label=testl,caption=Test]
int main(void) // main routine
{
int i, j; // Initialisation of counters

// The code below prints the 3x3 matrix
for (i=0; i<3; ++i) {
for (j=0; j<3; ++j) printf("%5.1f", m[i*3+j]);
putchar('\n');
}

cblas_dgemv(CblasRowMajor, CblasNoTrans,
3, 3, 1.0, m, 3, x, 1, 0.0, y, 1);

// The code below prints the 3x3 matrix - result of multiplication
for (i=0; i<3; ++i) printf("%5.1f\n", y[i]);

return 0;
}
\end{lstlisting}

\begin{lstlisting}[label=test2,caption=Test2]
int main(void) // main routine
{
	int i, j; // Initialisation of counters

	cout << i  << j;

	return 0;
}
\end{lstlisting}


\end{document}
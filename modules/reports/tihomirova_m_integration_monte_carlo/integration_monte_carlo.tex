\documentclass{report}

\usepackage[T2A]{fontenc}
\usepackage[utf8]{luainputenc}
\usepackage[english, russian]{babel}
\usepackage[pdftex]{hyperref}
\usepackage[14pt]{extsizes}
\usepackage{listings}
\usepackage{color}
\usepackage{geometry}
\usepackage{enumitem}
\usepackage{multirow}
\usepackage{graphicx}
\usepackage{indentfirst}
\usepackage{caption}
\usepackage{amsmath}

\geometry{a4paper,top=2cm,bottom=3cm,left=2cm,right=1.5cm}
\setlength{\parskip}{0.5cm}
\setlist{nolistsep, itemsep=0.3cm,parsep=0pt}

\lstset{language=C++,
		basicstyle=\footnotesize,
		keywordstyle=\color{blue}\ttfamily,
		stringstyle=\color{red}\ttfamily,
		commentstyle=\color{green}\ttfamily,
		morecomment=[l][\color{magenta}]{\#}, 
		tabsize=4,
		breaklines=true,
  		breakatwhitespace=true,
  		title=\lstname,       
}

\makeatletter
\renewcommand\@biblabel[1]{#1.\hfil}
\makeatother

\begin{document}

\begin{titlepage}

\begin{center}
Министерство науки и высшего образования Российской Федерации
\end{center}

\begin{center}
Федеральное государственное автономное образовательное учреждение высшего образования \\
Национальный исследовательский Нижегородский государственный университет им. Н.И. Лобачевского
\end{center}

\begin{center}
Институт информационных технологий, математики и механики
\end{center}

\vspace{4em}

\begin{center}
\textbf{\LargeОтчет по лабораторной работе} \\
\end{center}
\begin{center}
\textbf{\Large«Вычисление многомерных интегралов методом Монте-Карло»} \\
\end{center}

\vspace{4em}

\newbox{\lbox}
\savebox{\lbox}{\hbox{text}}
\newlength{\maxl}
\setlength{\maxl}{\wd\lbox}
\hfill\parbox{7cm}{
\hspace*{5cm}\hspace*{-5cm}\textbf{Выполнил:} \\ студент группы 381708-2 \\ Тихомирова М. А.\\
\\
\hspace*{5cm}\hspace*{-5cm}\textbf{Проверил:}\\ доцент кафедры МОСТ, \\ кандидат технических наук \\ Сысоев А. В.
}

\vspace{\fill}

\begin{center} Нижний Новгород \\ 2020 \end{center}

\end{titlepage}

\setcounter{page}{2}

\tableofcontents
\newpage

\section*{Введение}
\addcontentsline{toc}{section}{Введение}
Методами Монте-Карло называют численные методы решения математических задач при помощи моделирования случайных величин. Однако,
решать методами Монте-Карло можно любые математические задачи, а не
только задачи вероятностного происхождения, связанные со случайными
величинами.
\par Важнейшим приемом построения методов Монте-Карло явлеяется сведение задачи к расчету математических ожиданий. Так как математические ожидания чаще всего представляют собой обычные интегралы, то
центральное положение в теории метода Монте-Карло занимают методы
вычисления интегралов.
\par Преимущества недетерминированных методов особенно ярко проявляются при решении задач большой размерности, когда применение традиционных детерминированных методов затруднено или совсем невозможно.
\par Границы между простым и сложным, возможным и невозможным существуют всегда, но с развитием вычислительной техники сдвигаются вдаль. До появления электронных вычислительных машин (ЭВМ) методы МонтеКарло не могли стать универсальными численными методами, ибо моделирование случайных величин вручную — весьма трудоемкий процесс. Развитию методов Монте-Карло способствовало бурное развитие ЭВМ. Алгоритмы Монте-Карло сравнительно легко программируются и позволяют производить расчеты во многих задачах, недоступных для классических численных методов. Так как совершенствование ЭВМ продолжается,
есть все основания ожидать дальнейшего развития методов Монте-Карло
и дальнейшего расширения области их применения.
\par Цель работы – изучение применения метода Монте-Карло в решении задач численного интегрирования; cоздание программы для приближенного вычисления интегралов с использованием метода Монте-Карло и различных технологий для выполнения параллельных вычислений.
\newpage

\section*{Постановка задачи}
\addcontentsline{toc}{section}{Постановка задачи}
В рамках данной лабораторной работы ставится задача: реализовать метод Монте-Карло для вычисления многомерных интегралов с использованием различных технологий для выполнения параллельных вычислений.
\par Решение данной задачи разбивается на несколько подзадач:
\begin{itemize}
\item Реализовать последовательный алгоритм вычисления многомерных интегралов методом Монте-Карло.
\item Реализовать параллельный алгоритм вычисления многомерных интегралов методом Монте-Карло, используя OpenMP.
\item Реализовать параллельный алгоритм вычисления многомерных интегралов методом Монте-Карло, используя TBB.
\item Для каждой из реализаций провести вычислительные эксперименты и проанализировать результаты.
\end{itemize}

\newpage

\section*{Метод решения}
\addcontentsline{toc}{section}{Метод решения}
Пусть функция $y = f(x_{1}, ..., x_{m})$ непрерывна в ограниченной замкнутой области $G$ и требуется вычислить $m$-кратный интеграл $I$ по области $G$:
\begin{equation}I={\int...\int\limits_G} {f(x_{1}, ..., x_{m})dx_{1}...dx_{m}}.\end{equation}
\par Геометрически число $I$ представляет собой $(m + 1)$-мерный объем вертикального цилиндрического тела в пространстве $Ox_{1}x_{2} . . . x_{m}y$, построенного на основании $G$ и ограниченного сверху данной поверхностью $y = f(x)$,
где $x = (x1, . . . , xm)$.
\par Преобразуем интеграл так, чтобы новая область интегрирования $\omega$ целиком содержалась внутри единичного $m$-мерного куба. Пусть область интегрирования $G$ расположена в $m$-мерном параллелепипеде $a_{k} \leq x_{k} \leq b_{k}
(k = 1, . . . , m)$.
\par Сделаем замену переменных:
\begin{equation}x_{k} = a_{k} + (b_{k} - a_{k})\xi_{k} (k = 1, . . . , m).\end{equation}
\par Тогда m-мерный параллелепипед преобразуется в $m$-мерный единичный куб $0 \leq \xi_k \leq 1(k = 1, . . . , m)$, и, следовательно, новая область интегрирования $\omega$ будет целиком расположена внутри этого единичного куба.
\par Вычислим якобиан преобразования:
\begin{equation}\frac{D(x_{1}, ..., x_{m})}{D(\xi_{1}, ..., \xi_{m})} = 
\begin{vmatrix} 
b-a & 0 & ...& 0\\
0 & b-a & ... & 0\\
... & ... & ... & ...\\
0 & 0 & ... & b - a
\end{vmatrix}
= (b_{1} - a_{1})(b_{2} - a_{2}). . .(b_{m} - a_{m}).
\end{equation}
\par Таким образом,
\begin{equation}I = (b_{1} - a_{1})(b_{2} - a_{2}). . .(b_{m} - a_{m})J,\end{equation}
\par где
\begin{equation}J = {\int...\int\limits_\Omega}f[a_{1} + (b_{1} - a_{1})\xi_{1}, ... , a_{m} + (b_{m} - a_{m})\xi_m]d\xi_{1} ... d\xi_{m}\end{equation}
\par и область интегрирования $\Omega$ содержится внутри $m$-мерного единичного куба
\begin{equation}0 \leq \xi_{k} \leq 1 (k = 1, ... , m).\end{equation}
\par Пусть
\begin{equation}F(\xi_{1}, ... , \xi_{m}) = f[a_{1} + (b_{1} - a_{1})\xi_{1}, ... , a_{m} + (b_{m} - a_{m})\xi_{m}].\end{equation}
\par Выберем N равномерно распределенных на отрезке $[0, 1]$ последовательностей случайных чисел:
\begin{equation}
\begin{matrix} 
\xi_{1}^{(1)}, & \xi_{2}^{(1)}, & ..., & \xi_{m}^{(1)};\\
\xi_{1}^{(2)}, & \xi_{2}^{(2)}, & ..., & \xi_{m}^{(2)};\\
... & ... & ... & ...\\
\xi_{1}^{(N)}, & \xi_{2}^{(N)}, & ..., & \xi_{m}^{(N)}.
\end{matrix}
\end{equation}
\par Рассмотрим случайные точки $M_{i}(\xi_{1}^{(i)}, \xi_{2}^{(i)}, ..., \xi_{m}^{(i)})$. Выбрав достаточно большое $N$ число точек $M_{1}, . . . , M_{N}$ проверяем, какие из них принадлежат
области $\Omega$. Пусть $n$ точек принадлежат области $\Omega$.
\par Заметим, что относительно границы области $\Omega$ следует заранее договориться, причисляются ли граничные точки, или часть их, к области $\Omega$, или не причисляются к ней. В общем случае при гладкой границе это не имеет существенного значения; в отдельных случаях нужно решать вопрос с учетом конкретной обстановки.
\par Взяв достаточно большое n точек $M_{i}$, $F_{cp}$ приближенно можно положить равным:
\begin{equation}F_{cp} = \frac{1}{n} \sum_{i=1}^{n}{F(M_{i})} \end{equation}
\par откуда оценка интеграла $J$ выражается формулой 
\begin{equation}J = F_{cp} \Omega ,\end{equation}
\par где под $\Omega$ понимается $m$-мерный объем области интегрирования $\Omega$. Если вычисление объема $\Omega$ затруднительно, то можно принять $\Omega = \frac {n}{N}$ (из определения геометрической вероятности).
\par Таким образом, имеем оценку $I$ искомого интеграла $I$:
\begin{equation}I = V F_{cp} \Omega ,\end{equation}
\par где $V = (b_{1} - a_{1})...(b_{m} - a_{m})$ – объем параллелепипеда, ограничивающего исходную область интегрирования $G$. Отсюда получаем
\begin{equation}I = \frac{V}{N}\sum_{i=1}^{n}{F(M_{i})} \end{equation}
\par В частном случае, когда $\Omega$ есть единичный куб, проверка принадлежности точек $M_{i}$ области $\Omega$ становится излишней, то есть $n = N$ и мы имеем
\begin{equation}I = \frac{V}{N}\sum_{i=1}^{N}{F(M_{i})} \end{equation}

\newpage

\section*{Схема распараллеливания}
\addcontentsline{toc}{section}{Схема распараллеливания}
Для сходимости метода Монте-Карло критично качество генератора псевдослучайных чисел (ГПСЧ). Необходимо чтобы каждый поток получал последовательность чисел отличную от последовательностей полученных другими потоками, а также мог сгенерировать ее без взаимодействия с другими потоками. ГПСЧ должен обладать свойством масштабируемости: количество различных генерируемых последовательностей не должно быть привязано к количеству потоков.
\par В качестве генератора псевдослучайных чисел был выбран \emph{mt19937( )}, который при создании будет инициализироваться различными значениями в каждом потоке, что позволит генерировать различные последовательности. 
\par Общая схема работы параллельного алгоритма:
\begin{enumerate}
\item В каждом потоке создать экземпляр генератора псевдослучайных чисел и проинициализировать его уникальным значением.
\item В каждом потоке вычислить сумму значений функции для случайно сгенерированных точек. Количество точек равно порции вычислений для данного потока.
\item В мастер потоке просуммировать значения, полученные на каждом отдельно взятом потоке. Вычислить выборочное среднее и умножить на меру области интегрирования.
\end{enumerate}
\newpage

\section*{Описание программной реализации}
\addcontentsline{toc}{section}{Описание программной реализации}
\subsection*{OpenMP}
\addcontentsline{toc}{subsection}{OpenMP}
\par В OpenMP создание потоков происходит при достижении директивы \verb|#pragma omp| \verb|parallel|. Весь код, который будет выполняться параллельно, должен быть расположен внутри этой директивы.
\par OpenMP предоставляет специальную директиву \verb|for|, которая позволяет распараллеливать циклы, разбиение итераций между потоками происходит в зависимости от параметра \verb|schedule| (по-умолчанию static). Параметр \verb|reduction| позволяет определить список переменных, для которых результат вычислений в отдельных потоках будет собран в мастер-потоке.

\subsection*{TBB}
\addcontentsline{toc}{subsection}{TBB}
Библиотека TBB также предоставляет специальный функцию для выполнения редукции \verb|tbb::parallel_reduce|, которая в качестве обязательных параметров принимает:
 \begin{enumerate}
\item Итерационное пространство \verb|tbb::blocked_range|  ~--- в текущей задаче одномерное итерационное пространоство, которое задает диапазон в виде
полуинтервала [begin, end), где begin = 0, end = N (количество точек)
\item Лямбда-функция ~--- тело цикла for, реализует метод интегрирования. В потоке для случайно сгенерированных точек вычисляется сумма значений функции, и сохраняется это начение в локальную сумму.
\end{enumerate}
Разделение на порции вычислений происходит автоматически благодаря планировщику TBB.


\newpage

\section*{Подтверждение корректности}
\addcontentsline{toc}{section}{Подтверждение корректности}
Для подтверждения корректности в программе реализован набор тестов, разработанных при помощи библиотеки для модульного тестирования Google C++ Testing Framework. Проверяются случаи вычисления одномерных, двумерных и трехмерных интегралов. Значения полученные при вычислении методом Монте-Карло сравниваются со значениями полученным аналитически.
\par Успешное прохождение всех тестов является подтвержением корректной работы программы. 

\newpage

\section*{Результаты экспериментов}
\addcontentsline{toc}{section}{Результаты экспериментов}
Конфигурация системы:
\begin{itemize}
\item Процессор: Intel(R) Celeron(R) CPU N2840 @2.16GHz 2.16GHz
\item Число ядер: 2
\item Оперативная память: 4,00Гб
\item ОС: Windows 10
\end{itemize}

\par Эксперименты проводяться для \verb|10 000 000| случайных точек. 
\par Результаты экспериментов представлены в Таблице 1.

\begin{table}[!h]
\caption{Резултаты вычислительных экспериментов}
\centering
\begin{tabular}{|c|c|c|c|c|c|}
\hline
\multirow{3}{*}
	{\begin{tabular}[c]{@{}c@{}}Кол-во\\ потоков\end{tabular}} & 
\multirow{2}{*}
	{\begin{tabular}[c]{@{}c@{}}Последовательный\\ алгоритм\end{tabular}} & 
\multicolumn{4}{c|}
	{Параллельный алгоритм}	\\ 
	\cline{3-6} & & 
	\multicolumn{2}{c|}{OpenMP} & 
	\multicolumn{2}{c|}{TBB}
	\\ \cline{2-6}
	& t, с	    & t, с & speedup		& t, с & speedup		\\ \hline
2   & 1.73     & 1.09 & 1.587       	& 1.07 & 1.616           \\ \hline
\end{tabular}
\end{table}

\par По данным, полученным в результате экспериментов, можно сделать вывод, что параллельный алгоритм работает быстрее, чем последовательный. Так как количество ядер 2, нет необходимости рассматривать при большем количестве потоков.
\newpage

\section*{Заключение}
\addcontentsline{toc}{section}{Заключение}
В процессе выполнения лабораторной работы был изучен метод вычисления многомерных интегралов Монте-Карло и реализованы программы, последовательной и параллельных версий алгоритма с использованием OpenMP, TBB.
\par Разработаны тесты, подтверждающие корректность работы программы.
\par В заключение проведены эксперименты, подтверждающие эффективность параллельных версий алгоритма. По результатам опытов лучшей технологией для реализации параллельного метода Монте-Карло для вычисление многомерных интегралов оказалась TBB.
\newpage

\begin{thebibliography}{1}
\addcontentsline{toc}{section}{Список литературы}
\bibitem{Sysoev} Сысоев А.В., Мееров И.Б., Свистунов А.Н., Курылев А.Л., Сенин А.В., Шишков А.В., Корняков К.В., Сиднев А.А. «Параллельное программирование в системах с общей памятью. Инструментальная поддержка». Учебно-методические материалы по программе повышения квалификации «Технологии высокопроизводительных вычислений для обеспечения учебного процесса и научных исследований». Нижний Новгород, 2007, 110 с. 
\bibitem{Sobol} Соболь И.М. Численные методы Монте-Карло. - М.:Наука, 1973г. 312 стр. с илл.
\bibitem{Barkalov} Баркалов К.А. Методы параллельных вычислений. Н. Новгород: Изд-во Нижегородского госуниверситета им. Н.И. Лобачевского, 2011
\end{thebibliography}
\newpage

\section*{Приложение}
\addcontentsline{toc}{section}{Приложение}
\centerline{\bfseries Исходный код.} 

\lstinputlisting[language=C++, caption=Последовательная версия. Заголовочный файл]{../../../../modules/task_1/tihomirova_m_integration_monte_carlo/integration_monte_carlo.h}
\lstinputlisting[language=C++, caption=Последовательная версия. Cpp файл]{../../../../modules/task_1/tihomirova_m_integration_monte_carlo/integration_monte_carlo.cpp}
\lstinputlisting[language=C++, caption=Последовательная версия. Тесты]{../../../../modules/task_1/tihomirova_m_integration_monte_carlo/main.cpp}

\lstinputlisting[language=C++, caption=OpenMP версия. Заголовочный файл]{../../../../modules/task_2/tihomirova_m_integration_monte_carlo/integration_monte_carlo.h}
\lstinputlisting[language=C++, caption=OpenMP версия. Cpp файл]{../../../../modules/task_2/tihomirova_m_integration_monte_carlo/integration_monte_carlo.cpp}
\lstinputlisting[language=C++, caption=OpenMP версия. Тесты]{../../../../modules/task_2/tihomirova_m_integration_monte_carlo/main.cpp}

\lstinputlisting[language=C++, caption=TBB версия. Заголовочный файл]{../../../../modules/task_3/tihomirova_m_integration_monte_carlo/integration_monte_carlo.h}
\lstinputlisting[language=C++, caption=TBB версия. Cpp файл]{../../../../modules/task_3/tihomirova_m_integration_monte_carlo/integration_monte_carlo.cpp}
\lstinputlisting[language=C++, caption=TBB версия. Тесты]{../../../../modules/task_3/tihomirova_m_integration_monte_carlo/main.cpp}


\end{document}
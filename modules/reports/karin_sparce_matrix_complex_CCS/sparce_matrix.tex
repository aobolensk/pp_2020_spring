\documentclass{report}

\usepackage[T2A]{fontenc}
\usepackage[utf8]{luainputenc}
\usepackage[english, russian]{babel}
\usepackage[pdftex]{hyperref}
\usepackage[14pt]{extsizes}
\usepackage{listings}
\usepackage{color}
\usepackage{geometry}
\usepackage{enumitem}
\usepackage{multirow}
\usepackage{graphicx}
\usepackage{indentfirst}

% Устанавливаю поля, отступы между абзацами, отступы в списке
\geometry{a4paper,top=2cm,bottom=3cm,left=2cm,right=1.5cm}
\setlength{\parskip}{0.5cm}
\setlist{nolistsep, itemsep=0.3cm,parsep=0pt}

% Замена [1] -> 1. в списке литературы
\makeatletter
\renewcommand\@biblabel[1]{#1.\hfil}
\makeatother

\begin{document}

% Титульный лист
\begin{titlepage}

\begin{center}
Министерство науки и высшего образования Российской Федерации
\end{center}

\begin{center}
Федеральное государственное автономное образовательное учреждение высшего образования \\
Национальный исследовательский Нижегородский государственный университет им. Н.И. Лобачевского
\end{center}

\begin{center}
Институт информационных технологий, математики и механики
\end{center}

\vspace{3em}

\begin{center}
\textbf{\Large Отчет по лабораторной работе} \\
\end{center}
\begin{center}
\textbf{\Large «Умножение разреженных матриц. Элементы комплексного типа. Формат хранения - строковый (CCS)»} \\
\end{center}

\vspace{4em}

\newbox{\lbox}
\savebox{\lbox}{\hbox{text}}
\newlength{\maxl}
\setlength{\maxl}{\wd\lbox}
\hfill\parbox{7cm}{
\hspace*{5cm}\hspace*{-5cm}\textbf{Выполнил:} \\ студент группы 381706-1 \\ Карин Т. А.\\
\\
\hspace*{5cm}\hspace*{-5cm}\textbf{Проверил:}\\ доцент кафедры МОСТ, \\ кандидат технических наук \\ Сысоев А. В.
}

\vspace{\fill}

\begin{center} Нижний Новгород \\ 2020 \end{center}

\end{titlepage}
% Конец титульного листа

\setcounter{page}{2}

% Содержание
\tableofcontents
\newpage

% Введение
\section*{Введение}
\addcontentsline{toc}{section}{Введение}
Достаточно часто встречаются такие матрицы, в которых важными являютя лишь некоторые элементы. Матрицы, в которых большинство элементов являются нулевыми, называются разреженными. Хранить значения всех элементов таких матриц не рационально. Для экономии памяти их принято записывать в координатном, строковом (CRS) или столбцовом (CCS) форматах. В данной работе будет рассмотрен только столбцовый формат.
\par Для хранения матрицы в столбцовом формате необходимы 3 вектора:
\begin{enumerate}
	\item value. Хранит непосредственно ненулевые значения матрицы. Они располагаются последовательно: упорядочены по номеру столбца, а затем по номеру строки.
	\item Row. Каждый элемент хранит номер строки, в которой расположен соответствующий ему элемент вектора value.
	\item Col. Хранит номера элементов вектора value, с которых начинаются соответствующие строки матрицы. В отличие от предыдущих векторов, размер Col зависит от размера матрицы, а не от количества ненулевых элементов в ней.
\end{enumerate}
\par Одной из основных операций с матрицами является умножение. Так как способ хранения матрицы отличается от обычного, то и алгоритм умножения тоже немного видоизменён. Целью работы является написание алгоритма умножения разреженных матриц, хранящихся в столбцовом формате, и примененение к нему различных методов распараллеливания.
\newpage
% Конец введения

% Постановка задачи
\section*{Постановка задачи}
\addcontentsline{toc}{section}{Постановка задачи}
В ходе работы необходимо выполнить следующие задачи:
\begin {itemize}
	\item Реализовать последовательный алгоритм умножения разреженных матриц форамата CCS;
	\item Реализовать параллельный алгоритм умножения разреженных матриц с помощью технологии OpenMP;
	\item Написать параллельный алгоритм умножения разреженных матриц с помощью библиотеки TBB;
	\item Написать параллельный алгоритм умножения разреженных матриц, используя std::thread;
	\item Проверить корректность и эффективность полученных алгоритмов.
\end {itemize}
\newpage
% Конец постановки задачи

% Метод решения
\section*{Метод решения}
\addcontentsline{toc}{section}{Метод решения}

\newpage
% Конец метода решения

% Схема распараллеливания
\section*{Схема распараллеливания}
\addcontentsline{toc}{section}{Схема распараллеливания}

\newpage
% Конец схемы распараллеливания

% Описание программной реализации
\section*{Описание программной реализации}
\addcontentsline{toc}{section}{Описание программной реализации}


\subsection*{OpenMP}
\addcontentsline{toc}{subsection}{OpenMP}


\subsection*{TBB}
\addcontentsline{toc}{subsection}{TBB}


\subsection*{std::threads}
\addcontentsline{toc}{subsection}{std::threads}

\newpage
% Конец описания программной реализации

% Подтверждение корректности
\section*{Подтверждение корректности}
\addcontentsline{toc}{section}{Подтверждение корректности}

\newpage
% Конец подтверждения корректности


% Результаты экспериментов
\section*{Результаты экспериментов}
\addcontentsline{toc}{section}{Результаты экспериментов}


\newpage
% Конец результатов экспериментов


% Заключение
\section*{Заключение}
\addcontentsline{toc}{section}{Заключение}

\newpage
% Конец заключения



\end {document}
\documentclass{report}

\usepackage[T2A]{fontenc}
\usepackage[utf8]{luainputenc}
\usepackage[english, russian]{babel}
\usepackage[pdftex]{hyperref}
\usepackage[14pt]{extsizes}
\usepackage{listings}
\usepackage{color}
\usepackage{geometry}
\usepackage{enumitem}
\usepackage{multirow}
\usepackage{graphicx}
\usepackage{indentfirst}

\DeclareGraphicsExtensions{.jpg}

% Устанавливаю поля, отступы между абзацами, отступы в списке
\geometry{a4paper,top=2cm,bottom=3cm,left=2cm,right=1.5cm}
\setlength{\parskip}{0.5cm}
\setlist{nolistsep, itemsep=0.3cm,parsep=0pt}

% Мой стиль для листинга кода
\lstset{language=C++,
		basicstyle=\footnotesize,
		keywordstyle=\color{blue}\ttfamily,
		stringstyle=\color{red}\ttfamily,
		commentstyle=\color{green}\ttfamily,
		morecomment=[l][\color{magenta}]{\#}, 
		tabsize=4,
		breaklines=true,
  		breakatwhitespace=true,
  		title=\lstname,       
}

% Замена [1] -> 1. в списке литературы
\makeatletter
\renewcommand\@biblabel[1]{#1.\hfil}
\makeatother

\begin{document}

% Титульный лист
\begin{titlepage}

\begin{center}
Министерство науки и высшего образования Российской Федерации
\end{center}

\begin{center}
Федеральное государственное автономное образовательное учреждение высшего образования \\
Национальный исследовательский Нижегородский государственный университет им. Н.И. Лобачевского
\end{center}

\begin{center}
Институт информационных технологий, математики и механики
\end{center}

\vspace{4em}

\begin{center}
\textbf{\LargeОтчет по лабораторной работе} \\
\end{center}
\begin{center}
\textbf{\Large«Поразрядная сортировка для вещественных чисел с простым слиянием»} \\
\end{center}

\vspace{4em}

\newbox{\lbox}
\savebox{\lbox}{\hbox{text}}
\newlength{\maxl}
\setlength{\maxl}{\wd\lbox}
\hfill\parbox{7cm}{
\hspace*{5cm}\hspace*{-5cm}\textbf{Выполнил:} \\ студент группы 381706-2 \\ Мышкин А. А.\\
\\
\hspace*{5cm}\hspace*{-5cm}\textbf{Проверил:}\\ доцент кафедры МОСТ, \\ кандидат технических наук \\ Сысоев А. В.
}

\vspace{\fill}

\begin{center} Нижний Новгород \\ 2020 \end{center}

\end{titlepage}
% Конец титульного листа

\setcounter{page}{2}

% Содержание
\tableofcontents
\newpage

% Введение
\begin{center}\section*{Введение}\end{center}
\addcontentsline{toc}{section}{Введение}
\par Одна из самых популярных задач обработки данных является их сортировка. В основе сортировки лежит процесс упорядочивания наборов данных одного типа по возрастанию или убыванию значения какого-либо признака. В случае, когда элемент списка имеет несколько полей, поле, служащее критерием порядка, называется ключом сортировки. На практике в качестве ключа часто выступает число, а в остальных полях хранятся какие-либо данные, никак не влияющие на работу алгоритма.
\par В основном алгоритм сортировки является подзадачей и используется для обработки больших объёмов данных, поэтому необходимо чтобы этот процесс происходил максимально быстро и эффективно, тем самым оптимизировав выполнение всей поставленной задачи, в которой он используется.
\par В таком случае рассмотрим поразрядную сортировку вещественных чисел. Эта сортировка может похвастаться тем, что выполняет свою задачу за линейное время, то есть O(n), где n - количество элементов в массиве. Кроме последовательной реализации данного алгоритма будет представлено несколько вариантов параллельного алгоритма поразрядной сортировки.
\par Параллельные алгоритмы обеспечивают возможность параллельных вычислений, что на большом количестве данных может сильно ускорить время выполнения. В данной работе нам и предстоит узнать возможное ускорение алгоритма.
\newpage
% Конец введения

% Постановка задачи
\begin{center}\section*{Постановка задачи}\end{center}
\addcontentsline{toc}{section}{Постановка задачи}

В лабораторной работе предстоит выполнить несколько задач:

\begin{enumerate} 

\item Написать алгоритм поразрядной сортировки вещественных чисел
\item Реализовать последовательный алгоритм поразрядной сортировки и написать автоматические тесты для проверки работы алгоритма
\item Реализовать алгоритм параллельной поразрядной сортировки с простым слиянием, который позволит использовать технологии параллельного программирования 
\item Реализовать параллельный алгоритм, с использованием технологии OpenMP и в качестве проверки написать автоматические тесты
\item Реализовать параллельный алгоритм, с использованием технологии TBB и в качестве проверки написать автоматические тесты
\item Провести ряд экспериментов для анализа полученных результатов

\end{enumerate} 

\newpage
% Конец постановки задачи

% Описание алгоритма
\begin{center}\section*{Описание алгоритма}\end{center}
\addcontentsline{toc}{section}{Описание алгоритма}

\par Поразрядная сортировка сортирует элементы по разрядам чисел. В нашем случае сортировка будет LSD (least significant digit), то есть сортировка, которая начинается с младших разрядов. Для более эффективной реализации алгоритма следует за разряд считать байт числа. Так как у нас вещественные числа типа double, то число байт для каждого значения будет 8.
\par Сортировка протекает таким образом: начиная с младшего байта, проходимся по текущему байту всех чисел, записывая в группу по значению текущего разряда, затем элементы исходного массива перезаписываются в него с учетом их групп по значению. Но в виду представления чисел с плавающей запятой в памяти возникает проблема, в том что для старшего байта необходимо сделать особенную сортировку, так как в его страшем бите хранится знак числа (в случае если число положительное, то значение бита 0, иначе 1).
\par В данной реализации будет деление всего массива на массивы отрицательных и положительных чисел. В таком случае этот фактор не влияет на сортировку старшего байта. А после будет проведено слияние двух отсортированных массивов в общий.


\newpage
% Конец описания

% Схема распараллеливания
\begin{center}\section*{Схема распараллеливания}\end{center}
\addcontentsline{toc}{section}{Схема распараллеливания}
\subsection*{Технология OpenMP}
\addcontentsline{toc}{subsection}{Технология OpenMP}
\par Исходный массив разбивается по потокам, в случае если число элементов не кратно числу потоков, оставшиеся числа при делении числа элементов на число потоков, добавляются в нулевой поток.
\par Затем каждый поток вызывает поразрядную сортировку для своего выделенного массива.
\par В конце необходимо сделать слияние всех отсортированных массивов в один. В таком случае при отстутвии синхронизации процесса слияния массивов может возникнуть ошибка обращения некольких потоков к одному участку памяти в один момент времени. Во избежания ошибок данная операция делается в критической секции, которая обеспечивает синхронизацию между потоками.

\subsection*{Технология TBB}
\addcontentsline{toc}{subsection}{Технология TBB}
\par  Библиотека TBB предоставляет возможность писать параллельные программы на низком уровне – уровне «логических задач», работа с которыми, тем не менее, более удобна, чем напрямую с потоками.
\par Логическая задача предствляется в виде класса \verb|tbb::task|. Для написания пользовательских логических задач достаточно сделать наследование от \verb|tbb::task|.
\par Общая идея работы планировщика потоков библиотеки TBB заключается в следующем: каждый поток, созданный библиотекой, имеет свое множество (пул) готовых к выполнению задач. Это множество представляет собой динамический массив списков. Списки обрабатываются в порядке LIFO (last-in first-out).
\par Сначала заводится главная задача с помощью \verb|task::allocate_root()|, затем в главной задаче создаются подзадачи с помощью \verb|task::allocate_child()|. Все задачи начинает свое исполнения после \verb|task::spawn_root_and_wait(name)|, где \verb|name| - имя исполняемой задачи.
\par Алгоритм параллельной поразрядной сортировки.
\par Общий массив разделяется на массивы отрицательных и положительных элементов, для каждого создается главная задача. В этой задаче производится алгоритм сортировки подсчетом для каждого байта.
\par Алгоритм сортировки подсчетом состоит из следующих шагов:

\begin{enumerate} 

\item Для каждого потока создается массив подсчетов и заполняется нулями.
\item Каждый поток получает на обработку часть массива и выполняет подсчет элементов в свой массив подсчетов.
\item С помощью массива подсчетов со всех потоков выполняется вычисление смещений, по которым будут располагаться элементы при следующем проходе.
\item Каждый поток получает на обработку ту же часть массива, что и ранее,  и выполняет копирование элемента во вспомогательный массив по соответствующему индексу в массиве смещений.

\end{enumerate} 
\newpage
% Конец схемы распараллеливания

% Эксперименты
\begin{center}\section*{Эксперименты}\end{center}
\addcontentsline{toc}{section}{Эксперименты}
Конфигурация системы:
\begin{itemize}
\item Процессор: Intel(R) Core(TM) i7-3517U @ 1.90GHz 2.40GHz, 2 cores;
\item Оперативная память: 4Gb;
\item ОС: Microsoft Windows 10 Pro;
\end{itemize}
\par При проведении экспериментов задается случайный массив из \verb|n| элементов, пр параллельном исполнении число потоков равно 4. В итоге получены следующие результаты:
\begin{table}[h!]
\centering
 \begin{tabular}{||c c c c||} 
 \hline
 Число элементов & OpenMP & TBB & Seq \\ [0.5ex] 
 \hline\hline
 10 000 000 & 0.5193 & 0.4301 & 0.6318 \\ 
 15 000 000 &  0.8056 &  0.7152 & 0.9224 \\
 20 000 000 & 0.9947 &  0.8896 & 1.1392 \\
 25 000 000 & 1.3052 & 1.1883 & 1.5495 \\ [1ex] 
 \hline
 \end{tabular}
 \caption{Результаты экспериментов}
\end{table}
\par Из полученных результатов можно сделать несколько выводов:

\begin{enumerate} 
\item Благодаря параллельному исполнению время работы сортировки сократилось при использовании всех технологий.
\item Из-за малого количество ядер на тестируемой машине при параллельном исполнении время работы алгоритма уменьшается не более, чем на 30-33\%.
\item Благодаря эффективной работе планировшика потоков в библиотеке TBB, который оптимально распределяет задачи между свободными потоками, ускоряя
выполнение их задач, сокращает общее время работы относительно технологии OpenMP.
\item Сложность и затратность в выполнении такого этапа сортировки, как безопасное слияние отсортированных массивов, сильно увеличивает время выполнения сортировки.

\end{enumerate} 
 
\newpage
% Конец результатов экспериментов

% Заключение
\begin{center}\section*{Заключение}\end{center}
\addcontentsline{toc}{section}{Заключение}

Подводя итоги проделанной работы можно выделить несколько успешно выполненных задач:

\begin{itemize}
\item Разобран последовательный алгоритм поразрядной сортировки вещественных чисел.
\item Написана реализация поледовательного алгоритма поразрядной сортировки и автоматические тесты.
\item Написана реализация параллельного алгоритма сортировки с использованием технологии распараллеливания OpenMP.
\item Написана реализация параллельного алгоритма сортировки, основанной на логических задачах технологии TBB.
\item Проведены эксперименты двух технологий распараллеливания на больших объемах данных, сделан анализ по полученным результатам экспериментов.

\end{itemize} 

\newpage
% Конец заключения

% Список литературы
\begin{thebibliography}{1}
\addcontentsline{toc}{section}{Список литературы}

\bibitem{Gergel} Гергель В.П., Стронгин Р.Г. Основы параллельных вычислений для многопроцессорных вычислительных систем. Учебное пособие – Нижний Новгород: Изд-во ННГУ им. Н.И. Лобачевского, 2003. 184 с. ISBN 5-85746-602-4. 

\bibitem{Wiki1} Wikipedia: the free encyclopedia [Электронный ресурс] // URL: https://ru.wikipedia.org/wiki/Поразрядная\_сортировка

\bibitem{Online-article} Поразрядная сортировка // URL: http://algolist.ru/sort/radix\_sort.php

\bibitem{PDF1} Библиотека Intel Threading Building Blocks // URL: http://www.hpcc.unn.ru/multicore/materials/tech/tbb.pdf

\end{thebibliography}
\newpage
% Конец списка литературы

% Приложение
\begin{center}\section*{Приложение}\end{center}
\addcontentsline{toc}{section}{Приложение}
\subsection{Последовательная поразрядная сортировка}
\lstinputlisting[language=C++, caption=Заголовочный файл]{../../../../modules/task_1/myshkin_a_radix_sort_double/radix_sort_double.h}
\lstinputlisting[language=C++, caption=Cpp файл]{../../../../modules/task_1/myshkin_a_radix_sort_double/radix_sort_double.cpp}
\lstinputlisting[language=C++, caption=Тесты]{../../../../modules/task_1/myshkin_a_radix_sort_double/main.cpp}

\subsection{OpenMP}
\lstinputlisting[language=C++, caption=Заголовочный файл]{../../../../modules/task_2/myshkin_a_radix_sort_omp/radix_sort_double.h}
\lstinputlisting[language=C++, caption=Cpp файл]{../../../../modules/task_2/myshkin_a_radix_sort_omp/radix_sort_double.cpp}
\lstinputlisting[language=C++, caption=Тесты]{../../../../modules/task_2/myshkin_a_radix_sort_omp/main.cpp}

\subsection{TBB}
\lstinputlisting[language=C++, caption=Заголовочный файл]{../../../../modules/task_3/myshkin_a_radix_sort_double/radix_sort_tbb.h}
\lstinputlisting[language=C++, caption=Cpp файл]{../../../../modules/task_3/myshkin_a_radix_sort_double/radix_sort_tbb.cpp}
\lstinputlisting[language=C++, caption=Тесты]{../../../../modules/task_3/myshkin_a_radix_sort_double/main.cpp}
% Конец приложения

\end{document}

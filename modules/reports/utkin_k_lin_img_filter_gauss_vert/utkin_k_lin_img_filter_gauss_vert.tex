\documentclass{report}

% /////////////////////////////////////

\usepackage[14pt]{extsizes}
\usepackage{enumitem}
\usepackage[utf8]{luainputenc}
\usepackage[pdftex]{hyperref}
\usepackage{listings}
\usepackage{graphicx}
\usepackage{color}
\usepackage{geometry}
\usepackage[T2A]{fontenc}
\usepackage{multirow}
\usepackage[english, russian]{babel}
\usepackage{indentfirst}
\usepackage{tocloft}

\geometry{a4paper, top=2cm, bottom=3cm, left=2cm, right=1.5cm}

\setlength{\parskip}{0.5cm}
\setlist{nolistsep, itemsep=0.3cm, parsep=0pt}

\lstset{
	language=C++,
	basicstyle=\footnotesize,
	keywordstyle=\color{blue}\ttfamily,
	stringstyle=\color{red}\ttfamily,
	commentstyle=\color{green}\ttfamily,
	morecomment=[l][\color{magenta}]{\#}, 
	tabsize=4,
	breaklines=true,
	breakatwhitespace=true,
	title=\lstname,       
}

% /////////////////////////////////////

\begin{document}

	% /////////////////////////////

	\begin{titlepage}
	
		\begin{center}
			Министерство науки и высшего образования Российской Федерации
		\end{center}

		\begin{center}
			Федеральное государственное автономное образовательное учреждение высшего образования \\
			Национальный исследовательский Нижегородский государственный университет им. Н.И. Лобачевского
		\end{center}

		\begin{center}
			Институт информационных технологий, математики и механики
		\end{center}

		\vspace{\fill}

		\begin{center}
			\textbf{\LargeОтчет по лабораторной работе} \\
		\end{center}

		\begin{center}
			\textbf{\Large"Линейная фильтрация изображений (вертикальное разбиение). Ядро Гаусса 3x3"} \\
		\end{center}

		\vspace{\fill}

		\hfill\parbox{6cm}{
			\hspace*{5cm}\hspace*{-5cm}\textbf{Выполнил:} \\ студент группы 381708-1 \\ Уткин К. Д. \\
			\\
			\hspace*{5cm}\hspace*{-5cm}\textbf{Проверил:}\\ доцент кафедры МОСТ, \\ кандидат технических наук \\ Сысоев А. В.
		}

		\vspace{\fill}

		\begin{center}
			Нижний Новгород \\ 2020
		\end{center}

	\end{titlepage}

	% /////////////////////////////

	\renewcommand\contentsname{\Large{Содержание}}
	\renewcommand\thechapter{\arabic{chapter}}
	\setcounter{page}{2}
	\tableofcontents
	\newpage

	% /////////////////////////////

	\section*{Введение}
	\addcontentsline{toc}{section}{Введение}

	\par Под фильтрацией изображений понимают операцию, имеющую своим результатом изображение того же размера, полученное из исходного по некоторым правилам. Обычно интенсивность (цвет) каждого пикселя результирующего изображения обусловлена интенсивностями (цветами) пикселей, расположенных в некоторой его окрестности в исходном изображении. Фильтры — это и есть правила, которые задают фильтрацию
	\par Линейные фильтры представляют собой семейство фильтров, имеющих очень простое математическое описание. Вместе с тем они позволяют добиться самых разнообразных эффектов. Фильтр Гаусса — один из линейных фильтров. Гауссовская фильтрация является сглаживающей. Она эффективна при шумоподавлении и используется для размытия изображений.
	\par При фильтрации Гаусса каждый шаг (обработка пикселя) выполняется независимо от предыдущего, что позволяет увеличить скорость работы алгоритма при помощи технологий параллельных вычислений.

	\newpage

	% /////////////////////////////

	\section*{Постановка задачи}
	\addcontentsline{toc}{section}{Постановка задачи}

	\par Для выполнения лабораторной работы необходимо решить следующие задачи:

	\begin{enumerate}
		\item Изучить алгоритм фильтрации Гаусса;
		\item Реализовать последовательный алгоритм фильтрации Гаусса;
		\item Изучить технологии OpenMP и применить их для реализации параллельного алгоритма фильтрации Гаусса;
		\item Изучить технологии TBB и применить их для реализации параллельного алгоритма фильтрации Гаусса;
		\item Изучить технологии std::thread (язык C++) и применить их для реализации параллельного алгоритма фильтрации Гаусса;
		\item Для проверки корректности алгоритма необходимо: реализовать автоматические тесты при помощи Gtest и визуализировать результат работы алгоритма при помощи OpenCV;
		\item Провести эксперименты по времени работы и сравнить их результаты.
	\end{enumerate}

	\newpage

	% /////////////////////////////

	\section*{Описание алгоритма}
	\addcontentsline{toc}{section}{Описание алгоритма}
	
	\par Матрица свертки, ядро — это матрица коэффициентов, которая "умножается" на значение пикселей изображения для получения требуемого результата.
	\par Гауссовский фильтр имеет ненулевое ядро бесконечного размера. Однако ядро фильтра очень быстро убывает к нулю при удалении от точки (0, 0), и потому на практике можно ограничиться сверткой с окном небольшого размера вокруг (0, 0). В рамках данной лабораторной работы ограничимся ядром размером 3х3.
	\par Для заполнения матрицы (ядра) значениями воспользуемся функцией Гаусса:

	\begin{equation}
		F_{gauss}(i, j) = \frac{1}{2\pi\sigma^{2}}exp\left(-\frac{i^{2}+j^{2}}{2\sigma^{2}}\right)
	\end{equation}

	\par Данная функция позволяет заполнить ядро нормированными коэффициентами.
	\par Алгоритм фильтрации: пусть на текущем шаге выбран конкретный пиксель, необходимо применить матрицу свертки с выбранными выше коэффициентами — умножить RGB значения соседних пикселей на соответствующие коэффициенты в ядре. Новое значения для выбранного пикселя — сумма этих произведений.
	\par Очевидно, что возникают трудности с пикселями на границе — у них отсутствуют соседние пиксели . Для решения этой проблемы будем добавлять к исходному изображению рамку, которая соответствует крайним пикселям. Обрабатывать будем именно изображение в рамке.
	\par Таким образом получаем следующий набор функций для реализации базового алгоритма фильтрации Гаусса:

	\begin{enumerate}
		\item Функция добавления рамки к исходному изображению;
		\item Функция заполнения двумерного массива случайными значениями в диапазоне от 0 до 255 (RGB) — необходимо для реализации автоматических тестов;
		\item Функция фильтрации изображения.
	\end{enumerate}

	\newpage

	% /////////////////////////////

	\section*{Схема распараллеливания}
	\addcontentsline{toc}{section}{Схема распараллеливания}

	\par Значения пикселей изображения представим в виде одномерного массива, длина которого равна произведению ширины и высоты исходного изображения.
	\par Так как каждый шаг в фильтрации Гаусса не зависит от предыдущих, то задачу можно разбить на несколько частей, сгруппировав шаги любым образом. Для выполнения данной лабораторной работы воспользуемся вертикальным разбиением. В общем случае массив пикселей будет разбит на несколько блоков (одинаковой высоты — высоты изображения), число которых соответствует числу потоков. Размеры блоков зависят от множества факторов, поэтому они могут быть формально любыми, но на практике размеры получатся схожие.
	\newpage

	% /////////////////////////////

	\section*{Программная реализация}
	\addcontentsline{toc}{section}{Программная реализация}

	\subsection*{OpenMP}
	\addcontentsline{toc}{subsection}{OpenMP}
	\par Внутри цикла по ширине изображения перед вложенным циклом по высоте, используем директиву \verb|#pragma omp parallel for| — это позволит распределить выполнение цикла по столбцам изображения следующим образом: на каждой итерации первый освободившийся поток получает для обработки одномерный массив, длина которого соответствует высоте изображения;

	\subsection*{TBB}
	\addcontentsline{toc}{subsection}{TBB}
	\par Воспользуемся функцией \verb|tbb::parallel_for| для распределения работы по потокам. В качестве аргументов она принимает итерационное пространство и функтор. Для нашего случая, итерационное пространство — от 0 до числа, соответствующего числу столбцов изображения, а внутри функтора заложен алгоритм фильтрации Гаусса. При помощи технологий TBB массив со значениями пикселей изображения будет разбит на несколько блоков.

	\subsection*{std::threads}
	\addcontentsline{toc}{subsection}{std::threads}
	\par Сперва реализуем функцию \verb|threadFunction|. Суть её такова: она выполняется на каждом отдельном потоке, обрабатывая определенный участок изображения — блок, размеры которого она получает в качестве параметров. Разделив изображение по ширине на одинаковые блоки (последний блок может оказаться больше (шире) остальных), распределяем работу по потокам внутри цикла по числу потоков при помощи объекта \verb|std::thread| (он принимает в качестве параметра \verb|threadFunction|).

	\newpage

	% /////////////////////////////

	\section*{Результаты экспериментов}
	\addcontentsline{toc}{section}{Результаты экспериментов}
	\par Конфигурация системы:

	\begin{itemize}
		\item Процессор: AMD Ryzen 5 2600 3400 MHz (6 cores, 12 threads);
		\item Оперативная память: 16Gb DDR4 3000 MHz;
		\item ОС: Windows 10 Pro.
	\end{itemize}

	\par Для эксперимента возьмем изображение размером 5000x5000 пикселей и заполним его случайными значениями от 0 до 255. Проведем вычислительные эксперименты для каждой реализации алгоритма. Результаты получились следующими:
	\par Последовательный алгоритм (независимо от количества потоков) \\ Время работы: 7.558 с
	\par OpenMP (6 потоков) \\ Время работы: 1.687 с
	\par TBB (6 потоков) \\ Время работы: 1.486 с
	\par std::threads (6 потоков) \\ Время работы: 2.065 с

	\par Из результатов видно, что в среднем параллельный алгоритм на 6 потоках работает в 4.4 раза быстрее, чем последовательный. Лучший результат у технологии TBB, худший у std::threads.

	\newpage

	% /////////////////////////////

	\section*{Подтверждение корректности алгоритма}
	\addcontentsline{toc}{section}{Подтверждение корректности алгоритма}

	\par В соответствие с постановкой задач, для всех реализаций алгоритма фильтрации Гаусса были реализованы автоматически тесты, которые проверяют правильность выполнения всех функций.
	\par Также был реализован алгоритм (при помощи OpenCV), позволяющий визуализировать результат фильтрации:

	\begin{figure}[htbp]
		\centering
		\includegraphics{../../../../modules/reports/utkin_k_lin_img_filter_gauss_vert/source.png}
		\caption{исходное изображение}
	\end{figure}

	\begin{figure}[htbp]
		\centering
		\includegraphics{../../../../modules/reports/utkin_k_lin_img_filter_gauss_vert/filtered.png}
		\caption{изображение после фильтрации}
	\end{figure}

	\newpage

	% /////////////////////////////

	\section*{Заключение}
	\addcontentsline{toc}{section}{Заключение}

	\par При выполнении данной лабораторной работы, был подробно изучен алгоритм фильтрации Гаусса, который впоследствии был реализован в виде последовательного алгоритма и параллельного (с использованием различных технологий: OpenMP, TBB и std::threads). Была доказана эффективность параллельной реализации фильтрации посредством вычислительного эксперимента, в котором лучший результат показала технология TBB. Корректность алгоритма была проверена автоматическими тестами (gtest), а также была реализована визуализация работы алгоритма при помощи OpenCV.

	\newpage

	% /////////////////////////////

	\section*{Список литературы}
	\addcontentsline{toc}{section}{Список литературы}

	\begin{enumerate}
		\item Сысоев А.В., Мееров И.Б., Сиднев А.А. Средства разработки параллельных программ для систем с общей памятью. Библиотека Intel Threading Building Blocks. Учебно-методические материалы по программе повышения квалификации «Технологии высокопроизводительных вычислений для обеспечения учебного процесса и научных исследований». Нижний Новгород, 2007, 86 с.
		\item НОУ ИНТУИТ "Фильтрация изображений": \url{https://www.intuit.ru/studies/courses/993/163/lecture/4505};
		\item Habr "Матричные фильтры обработки изображений": \url{https://habr.com/ru/post/142818/};
		\item Intel® Threading Building Blocks Documentation: \url{https://software.intel.com/en-us/tbb-documentation};
		\item The OpenMP API specification for parallel programming: \url{http://www.openmp.org}.
	\end{enumerate}
	
	\newpage

	% /////////////////////////////

	\section*{Приложение}
	\addcontentsline{toc}{section}{Приложение}

	\subsection*{Последовательная версия}
	\addcontentsline{toc}{subsection}{Последовательная версия}
	\lstinputlisting[language=C++, caption=Заголовочный файл]{../../../../modules/task_1/utkin_k_lin_img_filter_gauss_vert/lin_img_filter_gauss_vert.h}
	\lstinputlisting[language=C++, caption=Реализация]{../../../../modules/task_1/utkin_k_lin_img_filter_gauss_vert/lin_img_filter_gauss_vert.cpp}
	\lstinputlisting[language=C++, caption=Тесты]{../../../../modules/task_1/utkin_k_lin_img_filter_gauss_vert/main.cpp}

	\subsection*{OpenMP}
	\addcontentsline{toc}{subsection}{Последовательная версия}
	\lstinputlisting[language=C++, caption=Заголовочный файл]{../../../../modules/task_2/utkin_k_lin_img_filter_gauss_vert/lin_img_filter_gauss_vert.h}
	\lstinputlisting[language=C++, caption=Реализация]{../../../../modules/task_2/utkin_k_lin_img_filter_gauss_vert/lin_img_filter_gauss_vert.cpp}
	\lstinputlisting[language=C++, caption=Тесты]{../../../../modules/task_2/utkin_k_lin_img_filter_gauss_vert/main.cpp}

	\subsection*{TBB}
	\addcontentsline{toc}{subsection}{Последовательная версия}
	\lstinputlisting[language=C++, caption=Заголовочный файл]{../../../../modules/task_3/utkin_k_lin_img_filter_gauss_vert/lin_img_filter_gauss_vert.h}
	\lstinputlisting[language=C++, caption=Реализация]{../../../../modules/task_3/utkin_k_lin_img_filter_gauss_vert/lin_img_filter_gauss_vert.cpp}
	\lstinputlisting[language=C++, caption=Тесты]{../../../../modules/task_3/utkin_k_lin_img_filter_gauss_vert/main.cpp}

	\subsection*{std::threads}
	\addcontentsline{toc}{subsection}{Последовательная версия}
	\lstinputlisting[language=C++, caption=Заголовочный файл]{../../../../modules/task_4/utkin_k_lin_img_filter_gauss_vert/lin_img_filter_gauss_vert.h}
	\lstinputlisting[language=C++, caption=Реализация]{../../../../modules/task_4/utkin_k_lin_img_filter_gauss_vert/lin_img_filter_gauss_vert.cpp}
	\lstinputlisting[language=C++, caption=Тесты]{../../../../modules/task_4/utkin_k_lin_img_filter_gauss_vert/main.cpp}

	\newpage

	% /////////////////////////////

\end{document}

% /////////////////////////////////////